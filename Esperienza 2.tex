\begin{figure}
    \centering
    \begin{circuitikz}[american, voltage shift=0.5]
    \ctikzset{bipoles/oscope/waveform=square}
    \draw
    (0,0)to [R,l=$R$,i=$i_R$,v=$V_R$] (4,0)
    to [C,l=$C$,v=$V_C$](4,-3)
    to [short,-](0,-3)
    (4,0) to [short,-](6,0)
    to [rmeterwa,t=V](6,-3)
    to [short,-](4,-3)
    (0,0) to [oscope,v_=$V_g$](0,-3);
\end{circuitikz}
    \caption{Circuito}
    \label{fig: Circuito RC}
\end{figure}
\section{Esperienza 2}
In questa esperienza di laboratorio si è cercato di comprendere al meglio il comportamento di elementi passivi come condensatori e resistenze interfacciandosi con strumenti di laboratorio come generatori di funzione ed oscilloscopi.
In particolare il circuito che maggiormente si è preso in esame è quello rappresentato nella figura \ref{fig: Circuito RC}.Si è inoltre cercato di capire l'errore che gli esperimenti di laboratorio portano a commettere se utilizzati inconsapevolmente.
\subsection{2.1}
Come prima esperienza si è cercato di stimare la costante di tempo $\tau$. Quest'ultima è definita a partire dalla risposta della differenza di potenziale ai capi della capacità. Questa costante può essere stimata sia durante la carica che la scarica del condensatore. Per simulare in laboratorio questo aspetto si è impostato il potenziale ai capi del genearatore come un onda quadra.
\subsection{Scarica del consensatore $R=15k\Omega $ $C=1nF$}
La $\tau$ teorica in questo caso può essere calcolata tramite la formula $\tau=RC$, pertanto in questo caso $\tau=15\mu s$. Dalla teoria dell'elettromagnetismo la forma teorica che dovrebbe avere il potenziale ai capi di $C$ vale:
\begin{equation}
\label{eq 1}
    V_C(t)=V_{C0}e^{-t/\tau}
\end{equation}
Ove $V_{C0}$ è la tensione iniziale ai capi $C$ che nel caso in questione è pari a $3V$.\\
Si è quindi fatto un fit dei dati sperimentali con la funzione \ref{eq 1} modificato con un parametro $t_0$ per accomodare al meglio il fit. Si è quindi usato il seguente modello:
\begin{equation}
    V_C(t)=V_{C0}e^{-(t-t_0)/\tau}
\end{equation}
Il risulato ottenuto è quello riportato in figura \ref{fig:Scarica RC1}
\begin{figure}
    \centering
    \includegraphics[trim=0 160 0 30, clip, width=\textwidth]{scarica_RC.png}
    \caption{Scarica RC}
    \label{fig:Scarica RC1}
\end{figure}
\subsection{Propagazione dell'errore}
Dalla teoria della propagazione dell'errore e dall'equazione $\tau=RC$ si ha:
\begin{equation}
    \frac{\Delta\tau}{\tau}=\frac{\Delta R}{R}+\frac{\Delta C}{C}
\end{equation}
Si è stimato l'errore su $C$ come pari al $20\%$ mentre la resistenza $R$ misurata con il multimetro è $14.96$\unit{\kohm} . Tramite il datasheet del multimetro si è stimato un $\Delta R = 0.15 $\unit{\kohm}. Pertanto:
\begin{equation}
    \Delta \tau = 3\unit{\us}
\end{equation}
Pertanto 
\begin{equation*}
    \tau=(15\pm 3) \unit{\us}
\end{equation*}
Eseguendo un fit dei dati tramite la relazione \ref{eq 1} si è ottenuta $\tau=18\unit{\us}$ che è all'interno del range della barra di errore.
\subsection{2.1.3}
Uno schema circuitale di questo tipo può essere adottato per eseguire semplici operazioni matematiche come l'integrazione e la derivazione.
Ad esempio misurando la misurando la tensione ai capi di $R$ è possibile osservare il comportamento derivatore del circuito, mentre misurando la tensione ai capi di $C$ si osserva il comportamento integratore del circuito.
Questo comportamento è stato osservato in laboratorio e i risultati sono stati riportati in figura \ref{fig:Derivatore e integratore}.
\begin{figure}[ht]
    \centering
    \begin{subfigure}[b]{0.5\textwidth}
        \centering
        \includegraphics[width=\textwidth]{RC_derivatore.png}
        \caption{Derivatore}
        \label{fig:Derivatore}
    \end{subfigure}%
    \hfill % adds horizontal space between the figures
    \begin{subfigure}[b]{0.5\textwidth}
        \centering
        \includegraphics[width=\textwidth]{RC_integratore.png}
        \caption{Integratore}
        \label{fig:Integratore}
    \end{subfigure}
    \caption{Comportamento del circuito come derivatore e integratore}
    \label{fig:Derivatore e integratore}
\end{figure}
\subsection{Scarica del consensatore $R=150$ \unit{\ohm} $C=100$\unit{\nF}}
Si è poi ripetuta la medesima esperienza, utilizzando però diversi valori per capacità e resistenza. Si è scelto il valore di $150 $\unit{\ohm} per la resistenza poichè, così facendo, essa diventa comparabile con la resistenza interna dell'oscilloscopio, del valore nominale di $50$\unit{\ohm}. \\
Nella prima parte dell'esperienza si è volutamente omessa questa considerazione poichè il rapporto $\frac{R_{g}}{R}$ è insignificante a tutti gli scopi pratici. 
La trattazione fisica del circuito è chiaramente la stessa del caso precedente, basta infatti considerare un circuito $RC$ equivalente con $R_{eq}$ pari a $ R + R_{g}=200\unit{\ohm}$ come nello schema circuitale \ref{scarica R 150}.\\
Nel calcolo della $\tau$ sarà pertanto necessario considerare la $R_{eq}$ ottenendo quindi:
\begin{equation*}
    \tau=(20\pm 3)\unit{\us}
\end{equation*}
Eseguendo il fit dei dati la $\tau$ misurata sperimentalmente è pari a $\tau=23\unit{\us}$
\begin{figure}
    \centering
    \begin{circuitikz}[american, voltage shift=0.5]
    \ctikzset{bipoles/oscope/waveform=square}
    \draw
    (0,0)to [R,l=$R_{g}$] (2,0) to[R,l=$R$] (5,0)
    to [C,l=$C$,v=$V_C$](5,-3)
    to [short,-](0,-3)
    (5,0) to [short,-](7,0)
    to [rmeterwa,t=V](7,-3)
    to [short,-](5,-3)
    (0,0) to [oscope,v_=$V_g$](0,-3);
    \draw[draw=black] (0.25,-0.8) rectangle ++(4.25,2);
    \node at (2.3,1.5){$R_{eq}$};
    \end{circuitikz}
    \label{scarica R 150}
    \caption{Schema circuitale}
\end{figure}
\begin{figure}
    \centering
    \includegraphics[trim=0 0 0 177, clip, width=\textwidth]{Scarica_RC.png}
    \caption{Caption}
    \label{fig:enter-label}
\end{figure}
Durante l'analisi si è notato che la tensione erogata dal generatore è deformata. Quest'osservazione si spiega ricordando che la misura della tensione viene effettuata a valle della resistenza del generatore che risulterà pari a 
\begin{equation}
    V_{mis}(t)=V_g-R_g I=V_g-R_g\frac{V_g-V_C(t)}{R_g+R}=\frac{R}{R_g+R}V_g+\frac{R_g}{R_g+R}V_C(t)
\end{equation}
Compare quindi un contributo dovuto alla tensione ai capi del condensatore che modifica la te nsione misurata sull'oscilloscopio. Chiaramente per $R>>R_g$ si ha $V_{mis}(t)\xrightarrow{}V_g$ che riconduce quindi al caso precedente. 
\subsubsection{Carica del condensatore}
Grazie all'ausilio del generatore d'onda quadra è stato possibile con una sola misura ricavare il grafico della carica del condensatore nei vari casi i cui grafici si possono trovare in figura \ref{fig:Carica condensatore}.
\begin{figure}
    \centering
    \includegraphics[scale=0.7]{Carica_RC.png}
    \caption{Carica condensatore}
    \label{fig:Carica condensatore}
\end{figure}
\begin{figure}
    \centering
    \begin{circuitikz}[american, voltage shift=0.5]
    \ctikzset{bipoles/oscope/waveform=square}
    \draw
    (0,0)to [R,l=$R_1$] (3,0)
    to [R,l=$R_2$,](3,-3) -- (0,-3);
    \draw (3,0) to [R,l=$R$](6,0)
    to [C,l=$C$](9,0)
    to[L,l=$L$,v=$v_L$](9,-3) -- (0,-3) to [oscope,v^<=$V_g$](0,0);
    \draw (9,0) -- (11,0) to [rmeterwa,t=V](11,-3) --(9,-3);
\end{circuitikz}
    \caption{Circuito}
    \label{fig: Circuito}
\end{figure}
\subsection{2.2}
Con quest'esperienza si cerca di capire il comportamneto in frequenza di un circuito $RC$.\\
In particolare si cerca di valutare il guadagno e lo sfasamento in funzione della frequenza.\\
Si cerca pertanto di ricavare il digramma di Bode del circuito.\\
\subsection{2.2.1 e 2.2.2}
Si è quindi montato il circuito come in figura \ref{fig: Circuito RC} e si è impostato il generatore di funzioni in modo tale che la tensione in ingresso fosse un'onda sinusoidale con tensione $5V_{pp}$.\\
Si è utilizzato $R=1.5$\unit{\kohm} e $C=1$\unit{\nF}.\\
Si è quindi fatta variare la frequenza e misurata la tensione ai capi del condensatore, e lo sfasamento tra la tensione in ingresso e di uscita ed eseguito un plot del guadagno e dello sfasamento in funzione della frequenza.\\
Il risultato ottenuto è riportato in figura \ref{fig: Guadagno RC}
\begin{figure}
    \centering
    \includegraphics[width=\textwidth]{Guadagno_RC.png}
    \caption{Guadagno RC}
    \label{fig: Guadagno RC}
\end{figure}
\subsection{2.2.3}
Abbiamo poi ripetuto le misure anche nel caso in cui $R=150$\unit{\ohm} e $C=100$\unit{\nF}.\\
Qui si riscontra lo stesso problema che si era ottenuto nella sezione 2.1.3.\\
E' stato quindi necessario considerare la resistenza interna dell'oscilloscopio.\\
Il risultato ottenuto è riportato in figura \ref{fig: Guadagno RC 150}
\begin{figure}
    \centering
    \includegraphics[width=\textwidth]{Guadagno_RC_150.png}
    \caption{Guadagno RC}
    \label{fig: Guadagno RC 150}
\end{figure}
\subsection{2.3}
Si è quindi passati a studiare il circuito in figura \ref{fig: Circuito}.\\
Si è passati quindi allo studio in frequenza di un circuito $RLC$\\
\subsubsection{2.3.1}
Come prima cosa si è cercato di valutare quali potessero essere i parametri caratteristici del circuito come $\alpha$ e la pulsazione di risonanza $\omega_0$.\\
